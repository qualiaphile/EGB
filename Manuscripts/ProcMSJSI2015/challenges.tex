
The following problem is not so much to prove something new, but rather to sharpen the methods of producing outputs to these algorithms.

\begin{problem}
Tractable termination of computation for the $x_{ij} \to t_i^3 t_j$ case.
\end{problem}

Also, in algebraic statistics.

\begin{problem}
Solve a big equivariant algebraic statistics problem.
\end{problem}


The following would advance the basic understanding of the structure of these special class of ideals.

\begin{problem}
What is the combinatorics in the general case for more than two indices?  (See \cite{KKL:equivariant-markov}).
\end{problem}

Some basic open question in the theory of equivariant Gr\"obner bases remain.

\begin{question}
Is there an equivariant Gr\"obner basis for this situation or is this only a finite generation result?  (See \cite{KKL:equivariant-markov, draisma2013noetherianity}).
\end{question}

In that regard, one can ask about the structure of all $P$-orders for equivariant Gr\"obner bases.  

\begin{question}
Is there an equivariant Gr\"obner fan?  (See \cite{sturmfels1996grobner}).
\end{question}

There are of course theoretical questions involving the structure of these generating sets.  For instance, here is a basic problem:

\begin{problem}
Study the infinite graph theoretical applications of equivariant Gr\"obner bases.  (See e.g. \cite{de2010recognizing}).
\end{problem}

More questions about toric ideals can also be asked.  For instance, as far as we know, it is still an open question whether every toric chain of ideals stabilizes (see \cite{Hillar13}).

\begin{question}
Does every toric chain of ideals stabilize?
\end{question}

\begin{question}
And what does representation theory says about its structure?  \cite{}
\end{question}


\begin{problem}
Prove formulas for structure constants of chains of ideals.  (See e.g. \cite{KKL:equivariant-markov, Hillar13, hillar2016corrigendum}).
\end{problem}


\begin{problem}
Determine computational complexity bounds for these algorithmic problems.  (See e.g. \cite{yap2000fundamental}).
\end{problem}


\begin{problem}
Develop a mapping from EGBs to geometric properties of infinite varieties.  (See e.g. \cite{Nagel, krone2016hilbert}).
\end{problem}

\begin{question}
What about other monoids?
\end{question}


\begin{question}
What about non-commutative extensions?  \cite{shirshov1962some, bokut1976embeddings, bergman1978diamond, la2009letterplace}
\end{question}



