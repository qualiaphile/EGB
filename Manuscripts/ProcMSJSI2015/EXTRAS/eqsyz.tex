\documentclass{amsart}
\usepackage{all}

\begin{document}
\title{Inc-equivariant syzygies}
\author{Jan Draisma, Robert Krone, and Anton Leykin}
\maketitle

\subsection*{Increasing maps}

Write $\NN=\{1,2,3,\ldots\}$ and, for $n \in \ZZ_{\geq 0}$, write
$[n]:=\{1,\ldots,n\}$. Let $\Pi$ denote the monoid of all increasing maps
$\pi:\NN \to \NN$ such that $\im(\pi)$ has a finite complement. For $i
\in \NN$ let $\rho_i$ denote the element of $\Pi$ defined by
\[ \rho_i(j)=
\begin{cases} 
	j & \text{if $j<i$, and}\\
	j+1 & \text{if $j \geq i$.}
\end{cases}
\]
The $\rho_i$ generate $\Pi$, and they satisfy the following fundamental
relations:
\[ \rho_{j+1}\rho_i=\rho_i \rho_j \text{ if }j \geq i. \]
In fact, this gives a presentation of $\Pi$, and any element of $\Pi$
has a unique expression of the form $\rho_{i_1} \cdots \rho_{i_d}$
with $d \in \ZZ_{\geq 0}$ and $i_1 \leq \ldots \leq i_d$.

\subsection*{The skew polynomial ring}

Fix $k \in \ZZ_{\geq 0}$. Set $R:=K[x_{ij} \mid i \in [k], j \in \NN]$,
where $K$ is a field.\footnote{Or do we want a Noetherian ring?}
Then we have an action of $\Pi$ on $R$ by $K$-algebra automorphisms
determined by $\pi(x_{ij})=x_{i\pi(j)}$. We construct the {\em skew
polynomial ring} $R*\Pi$, which is the free left $R$-module with basis
$\Pi$, equipped with a multiplication defined by $(r_1 \pi_1)\cdot(r_2
\pi_2)=(r_1\pi_1(r_2))(\pi_1\pi_2)$. An element of $R*\Pi$ of the form
$x^{\alpha} \pi$ with $x^\alpha \in R$ a (finite) monomial and $\pi \in
\Pi$ is called a {\em monomial} in $R$. From \cite{} we know that $R$
is a Noetherian left $R*\Pi$-module. We will prove a generalisation of
this fact.

\subsection*{Monomial orders}

Let $n$ be a nonnegative integer, and consider the quotient of the
left $R*\Pi$-module $R*\Pi$ by its submodule $J_n \subseteq R*\Pi$
defined as the left submodule generated by $\{\rho_i-1 \mid i>n\}$.
The quotient $M_n:=(R*\Pi/J_n)$ is a free $R$-module with $R$-basis
$\rho_I:=\rho_{i_1} \cdots \rho_{i_d}$, where $I=(i_1,i_2,\ldots,i_d)$
and $1 \leq i_1 \leq \cdots \leq i_d \leq n$. A {\em monomial} in $M_n$
is an element of the form $x^\alpha \rho_I$ with $x^\alpha$ a (finite)
monomial in the variables $x_1,x_2,\ldots$ and $I$ as before. These
form a $K$-basis of $M_n$. Similarly, for $\bn=(n_1,\ldots,n_p) \in
\ZZ_{\geq 0}$ we set $M_{\bn}:=M_{n_1} \times \cdots \times M_{n_p}$.
This is an $R*\Pi$-module with $K$-basis of monomials $uE_q$ where
$E_q\ (q=1,\ldots,p)$ is a standard basis vector and $u \in M_{n_q}$
is a monomial.

\begin{de}
A {\em monomial order} on $M_{\bn}$ is a well-order $<$ on monomials
with the property that if $u,v$ are monomials in $M_\bn$ and $m$ is a
monomial in $R*\Pi$, then $u<v$ implies $mu<mv$.
\end{de}

For example, define an order as follows. Write $u=x^\alpha \rho_I E_q$
and $v=x^\beta \rho_J E_r$ and set $u<v$ if either $q<r$ or ($q=r$ and
$\rho_I<\rho_J$) or ($q=r$ and $\rho_I=\rho_J$ and $x^\alpha < x^\beta$);
where for the last order we choose any monomial order on $R$ preserved by
$\Pi$ (this means $<$ is a monomial order and that $x^\alpha < x^\beta$
implies $\pi(x^\alpha)<\pi(x^\beta)$ for all $\pi \in \Pi$), and for the
second order we set $(i_1,\ldots,i_d)<(j_1,\ldots,j_e)$ if $d<e$ or if
both $d=e$ and the right-most non-zero entry of $(i_1-j_1,\ldots,i_d-j_d)$
is negative.

\begin{lm}
The order just defined is a monomial order on $M_{\bn}$. 
\end{lm}

\begin{proof}
Suppose that $u<v$ are as above. Certainly $x^\gamma u < x^\gamma
v$, since the chosen order on monomials in $x$ is a monomial order.
Furthermore, if $u$ ends in $E_q$ and $v$ ends in $E_r$ with $q<r$,
then we will have $m u < m v$ for all monomials $m \in R*\Pi$. So we are
left with the case where $q=r$, and we want to check that $\rho_l u <
\rho_l v$ for any $l \in \NN$. If $\rho_I=\rho_J$, then we know that
$x^\alpha<x^\beta$, and we find
\[ \rho_l u=
\rho_l x^\alpha \rho_I E_q=
\rho_l(x^\alpha) \rho_l \rho_I E_q<
\rho_l(x^\beta) \rho_l \rho_J E_q=
\rho_l v, \]
where we have used that the monomial order on the $x$-monomials is
preserved by $\Pi$. Finally, consider the case where $\rho_I<\rho_J$.
Then we claim that $\rho_l \rho_I < \rho_l \rho_J$, as well. To see this,
we note that, in $M_{n_q}$, we have
\[ \rho_l \rho_I=\rho_{i_1} \cdots \rho_{i_a} \rho_{l-a} \rho_{i_{a+1}}
\cdots \rho_{i_d}=\rho_{I'}, \]
where $a$ is the maximal element in $[d]$ for which $l-a \geq i_a$
or $0$ if no such element exists. Here $l-a$ may be strictly larger
than $n_q$. In that case, $a=d$ and the last term $\rho_{l-a}$ is to be
cancelled to get a normal form of a monomial in $M_{n_q}$; $\rho_l \rho_I$
then has degree $d$ instead of the expected $d+1$. Similarly, we have
\[ \rho_l \rho_J=\rho_{j_1} \cdots \rho_{j_b} \rho_{l-b}
\rho_{j_{b+1}} \cdots \rho_{j_e}=\rho_{J'}, \]
with the same remarks. Suppose first that $d<e$. If the degree of
$\rho_l \rho_I$ is $d+1$, then we have $l-d \leq n_q$ and hence also
$l-e \leq n_q$, so that the degree of $\rho_l \rho_J$ equals $e+1$.
If the degree of $\rho_l \rho_I$ is $d$, then this is certainly smaller
than the degree of $\rho_l \rho_J$, which equals $e$ or $e+1$. In either
case we find that $\rho_l \rho_I < \rho_l \rho_J$.

Next suppose that $d=e$. Then the degrees of $\rho_l \rho_I$
and $\rho_l \rho_J$ are both equal to $d$ (in which case $\rho_l
\rho_I=\rho_I<\rho_J=\rho_l \rho_J$) or both equal to $d+1$, namely,
when $l-d \leq n_q$. Then let $f \in [d]$ be the largest index with $i_f
\neq j_f$, so that by assumption $i_f<j_f$. If $f < a$, then $i$ and $j$
agree on the tails starting at $a$, and hence by the choice of $a$ and $b$
we have $a=b$. Then $f \in [d+1]$ is also the largest index where $I'$
and $J'$ differ, and we have $i'_f=i_f<j_f=j'_f$. Suppose that $f=a$.
Then $j_{a+1}=i_{a+1} > l-(a+1)$, so $b \leq a$. If $b=a$, then we are
in the previous case. If $b<a$, then we have $i'_{a+1}=l-a < j_a \leq
j_{a+1}=j'_{a+1}$, and $a+1$ is the largest index where $I'$ and $J'$
differ.  If $f>a$, then we have $l-f<i_f<j_f$, so also $f>b$. Then
$f+1$ is the largest index where $I'$ and $J'$ differ, and we have
$i'_{f+1}=i_f<j_f=j'_{f+1}$. In all cases, we find $\rho_l \rho_I <
\rho_l \rho_J$, as desired.
\end{proof}

\section*{Well-partial-orders}

Define the divisibility order $|$ on monomials in $M_\bn$ by $u|v$
if and only if there exists a monomial $m \in R*\Pi$ such that $mu = v$.

\begin{lm}
Any monomial order $<$ refines $|$. 
\end{lm}

\begin{proof}
Indeed, otherwise there would be monomials $u \in M_\bn$ and $m \in
R*\Pi$ such that $u>mu$. Multiplying both sides from the left with $m$
yields $mu>m^2u$, etc. So we find an infinite strictly decreasing chain
$u>mu>m^2u>\ldots$, which contradicts that $<$ is a well-order.
\end{proof}

\begin{prop}
The divisibility order on $M_\bn$ is a well-partial-order.
\end{prop}

The proof uses that there exists a monomial order $<$ on $M_\bn$, for example
as in the previous section.

\begin{proof}
If not, then there exists a {\em bad sequence} of monomials, which we may
assume all end on the same $E_q$, say, $E_1$: $u_1 E_1,u_2 E_1,\ldots$
of monomials, that is, a sequence in which $u_i \not | u_j$ for all
$i<j$. Moreover, we
can choose the sequence such that $u_1$ is $<$-minimal among all bad
sequences, and $u_2$ is minimal among all bad sequences starting with
$u_1$, {\em etc.} By Dickson's lemma, there exists an infinite subsequence
$u_{1'},u_{2'},\ldots$ in which the exponent vectors in $\ZZ_{\geq 0}^k$
of the first column of $x$ form a weakly increasing chain. Suppose
that this sequence has an infinite subsequence $u_{1''},u_{2''},\ldots$
in which the exponent of $\rho_1$ is at least $1$. Then we can write
\[ u_{i''}=(x^{\alpha_{i''}} \rho_1) \cdot v_{i''} \]
where $\alpha_{i''}$ is a $k \times \NN$-matrix with zeroes
everywhere except in the first column, and where the exponent of
$\rho_1$ in $v_{i''}$ is one less than in $u_{i''}$. In particular,
we have $v_{i''} \leq u_{i''}$ by the lemma, and moreover $v_{i''}
\neq u_{i''}$, so $v_{i''}<u_{i''}$. Then consider the sequence \[
u_1,u_2,\ldots,u_{1''-1},v_{1''},v_{2''},\ldots.\] This sequence is
smaller in that $u_{1''}$ has been replaced by $v_{1''}$, and to
arrive at a contradiction we claim that it is still bad sequence.
Indeed, clearly $u_i \not | u_j$ with $1 \leq i < j \leq 1''-1$. Also,
if $u_i | v_{j''}$, then certainly $u_i | u_{j''}$ since
$u_{j''}$ is
a multiple of $v_{j''}$. Hence $u_i \not | v_{j''}$. To complete the
proof of the claim, we need to show that $v_{i''} \not | v_{j''}$ for
$i<j$. Suppose the converse, so that $v_{j''}=x^\beta \rho_I v_{i''}$
for suitable $\beta$ and $I$. If $I=(i_1,\ldots,i_d)$, then write
$I':=(i_1'+1,\ldots,i'_d+1)$. Similarly, let $\beta'$ be the exponent
vector obtained from $\beta$ by shifting all columns one to the rights,
so that $\rho_1 x^\beta=x^{\beta'} \rho_1$. Then compute
\begin{align*} 
&(x^{\alpha_{j''}-\alpha_{i''}} x^{\beta'}
\rho_{I'})u_{i''}\\
&=(x^{\alpha_{j''}-\alpha_{i''}} x^{\beta'} \rho_{I'})
(x^{\alpha_{i''}} \rho_1) \cdot v_{i''}\\
&=x^{\alpha_{j''}-\alpha_{i''}} x^{\beta'} 
x^{\alpha_{i''}} \rho_{I'} \rho_1 \cdot v_{i''}\\
\intertext{(since $\alpha_{i''}$ has zeroes in all columns
affected by $\rho_{I'}$)}
&=x^{\alpha_{j''}-\alpha_{i''}} x^{\beta'} 
x^{\alpha_{i''}} \rho_1 \rho_I \cdot v_{i''}\\
&=x^{\alpha_{j''}} \rho_1 x^{\beta} 
\rho_I \cdot v_{i''}\\
&=x^{\alpha_{j''}} \rho_1 v_{j''}=u_{j''},
\end{align*}
a contradiction. 

If no infinite subsequence exists where the exponent of $\rho_1$ is at
least $1$, then pass to a subsequence where $\rho_1$ does not appear at
all and the exponent vectors corresponding to the {\em second} column
of $x$ form a weakly increasing chain, and repeat the argument with
$\rho_2$. Proceed until $\rho_n$.  If no infinite subsequence exists
where the $u_{i'}$ contain any of the $\rho_1,\ldots,\rho_{n_1}$, then we
have found a subsequence where the first $k \times n$-block of exponents
of $x$ increases weakly and in which the $\rho_1,\ldots,\rho_{n_1}$
do not appear at all. Then we can act with $\rho_{n+1}$ as in the usual
Cohen-Hillar-Sullivant argument, because it gets modded out in $M_\bn$
anyway.

\end{proof}

\section*{Noetherianity of $M^p$}

\begin{thm}
For any $p,n \in \ZZ_{\geq 0}$, the module $Mp=(R*\Pi/ J_n)^p$ is a
Noetherian $R*\Pi$-module.
\end{thm}

The special case where $p=1$ and $n=0$ is the aforementioned theorem.

\begin{proof}
Let $N$ be a submodule, and consider the set of $|$-minimal elements of
$\{\lmon(u) \mid u \in N\}$. By the proposition, this set is finite, ,
and by division with remainder, the corresponding finitely many $u \in N$
generate $N$.
\end{proof}

\end{document}
