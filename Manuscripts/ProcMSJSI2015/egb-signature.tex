Here we decribe an approach to computing equivariant \GBs\ utilizing the information stored in {\em signatures}. 

Signature-based algorithms for computing \GBs\ in the most common (finite-dimensional commutative) setting acquired popularity due to Faugere's F5 (see a short description in \S 4 of Chapter 10 of the new edition of Cox, Little, and O'Shea~\cite{Cox-Little-OShea:I-V-A}).      
We give a description of one of the signature-based approaches due to Gao {\em et al.} in \cite{Gao-Volny-Wang:signature-GBs} followed by its modification needed to compute equivariant \GBs.  

\subsection{Strong \GB}

Let $I=\ideal{F} \subset R=K[x_1,\ldots,x_n]$, where $|F|=r\in\NN$.

A subset $G$ of 
\[
P = \{(s,f)\in R^r\times R \mid f=s\cdot F = \sum_{i=1}^r s_iF_i\}.
\]
is called a {\em strong \GB } if every non-zero pair is {\em top-reducible} by some pair in $G$.

A pair $(s_f,f)$ is {\em top-reducible} by $(s_g,g)$ if $\LM g |\LM f$ and 
for some $a$, $\LM f = a \LM g$, we have $a\LM s_g\leq \LM s_f$.  If the reduction,
\[
(s_{f'},f') := (s_f-as_g,f-ag),
\]
has $\LM s_{f'} = \LM s_f$, then it is {\em regular top-reducible}.

If $G$ is a strong \GB, then by Proposition 2.2 of~\cite{Gao-Volny-Wang:signature-GBs}
\begin{enumerate}
   \item $\{f:(s,f)\in G\}$ is a \GB\ of $I$ and 
   \item $\{s:(s,0)\in G\}$ is a \GB\ of the module of syzygies $\Syz(F) \subset R^r$.
\end{enumerate}
   
Take two pairs $p_f=(s_f,f)$ and $p_g=(s_g,g)$. For monomials $a$ and $b$ such that $a\LM f = b\LM g \in \lcm(\LM f,\LM g)$, form a {\em J-pair} by taking the ``larger side'' of the corresponding S-polynomial: e.g., if $a\LM s_f \geq b\LM s_g$ then the J-pair is $(as_f,af)$.

We denote the set of all J-pairs of $p_f$ and $p_g$ as $J_{p_f,p_g}$. Note that $\lcm(\LM f,\LM g)$, the {\em set} of lowest common multiples, has one element in our current setting and so does $J_{p_f,p_g}$.  

\begin{example} If 
\begin{align*}
p_f &= (e_1+\ldots,\,x_1^2x_2+\ldots)\\
p_g &= (x_2e_1+\ldots,\,x_1x_2^2+\ldots)
\end{align*}
then, since $x_2\LM s_f < x_1\LM s_g$,
\[
J_{p_g,p_f} = \{x_1p_g\} = \{(x_1x_2e_1+\ldots, x_1^2x_2^2+\ldots)\}.
\]
\end{example}

A pair $(s_f,f)$ is {\em covered} by $(s_g,g)$ if $\LM g | \LM f$ and for some $a$ such that $\LM f = a \LM g$ we have $a\LM s_g < \LM s_f$. 

\begin{algorithm}[StrongBuchberger] \label{alg:StrongBuchberger}
\begin{algorithmic}[1]
\REQUIRE $F \subset R$.
\ENSURE $G \cup S$ is a strong \GB\ for $F$.
\smallskip \hrule \smallskip

\STATE $G\gets \emptyset$, $S\gets \emptyset$ 
\STATE $J\gets \{(e_i,F_i):i\in r=|F|\} \subset R^r\times R$ 
\WHILE{$J\neq\emptyset$}
	\STATE pick $p_f = (s_f,f) \in J$; $J\gets J\setminus\{p_f\}$
	\STATE $p_h=(s_h,h) \gets$ {\em regular top-reduction} of $(s_f,f)$ with respect to $G$
  	\IF{$h \neq 0$}
		\STATE $G\gets G\cup \{p_h\}$
		\STATE append to $J$ all $J$-pairs $\bigcup_{(p_g)\in G}J_{p_g,p_h}$ not {\em covered} by $G \cup S$ 
        \ELSE 
                \STATE $S\gets S\cup\{(s_h,0)\}$
	\ENDIF
\ENDWHILE
\smallskip \hrule \smallskip
\end{algorithmic}
\end{algorithm}

A proof of termination relies on {\em Noetherianity} of the free module $R^r$. 

\subsection{Translation to an equivariant setting}
(Main point: the direct translation is impossible!)


Let us go back to an infinite-dimensional polynomial ring $R=K[X]$ with some $\Pi$-action. As a running example take $R=K[x_i,\, i\in\NN]$ with a $\Pi$-compatible order, $\Pi=\Inc$.

To draw parallels with the approach of the previous section we need to work with pairs 
\[
P = \{(s,f)\in (R*\Pi)^r\times R \mid f=s\cdot F\}.
\]
Recall that, for instance, in our running example $\Pi=\{\rho_i:i\in\NN\} \subset \Inc$, so \[R*\Pi = K[X]*\Pi = K([X]*\Pi).\] 
The semidirect product $[X]*\Pi$ is a {\em non-Noetherian noncommutative} monoid. If follows that the (left) module $(R*\Pi)^r$ is not $\Pi$-Noetherian. 

A strong equivariant \GB, that can be defined similarly to the strong \GB\ in the previous section, is infinite (for) a nonzero $\Pi$-invariant ideal). 
For instance, for a $I = R = K[x_1,x_2,\ldots]$ has a \GB\ $\{1\}$.
However, a strong \GB\ has to include $\{(\rho_i-1)e_1 \mid i\in\NN \} \subset (R*\Pi)^1$. 

We found a way to modify Algorithm~\ref{alg:StrongBuchberger} to compute an equivariant \GB.  The algorithm below, of course, falls short of computing a strong equivariant \GB, but the partial information computed about the syzygies and the mechanism of top-reduction of J-pairs eliminate a large number of unnecessary iterations in a na\"ive implementation of an equivariant Buchberger's algorithm~(Algorithm \ref{alg:Buchberger}).

\begin{algorithm}[EquivariantStrongBuchberger]
\begin{algorithmic}[1]
\REQUIRE $F$ .
\ENSURE $G$ such that $\pi_2(G)$ is an equivariant \GB\ of $\ideal{F}_\Pi$.
\smallskip \hrule \smallskip
\STATE $r\gets |F|$.
\STATE $G\gets \emptyset$, $S\gets \emptyset$ 
\STATE $J\gets \{(e_i,F_i):i\in r=|F|\} \subset R^r\times R$ 
\WHILE{$J\neq\emptyset$}
	\STATE pick $p_f = (s_f,f) \in J$; $J\gets J\setminus\{p_f\}$
	\STATE $p_h=(s_h,h) \gets$ {\em regular top-reduction} of $(s_f,f)$ with respect to $G$
  	\IF{$h \neq 0$}
		\STATE \hl{$h' \gets\NF_{\Pi\pi_2(G)} h$}
	        \IF{$h' \neq 0$}
			\IF{\hl{$h' \neq h$}}
				\STATE \hl{$r\gets r+1$}
				\STATE \hl{$p_h \gets (e_r,h')$}
			\ENDIF
			\STATE $G\gets G\cup \{p_h\}$
			\STATE append to $J$ all $J$-pairs $\bigcup_{(p_g)\in G}J_{p_g,p_h}$ not {\em covered} by $G \cup S$ 
		\ENDIF
        \ELSE 
                \STATE $S\gets S\cup\{(s_h,0)\}$
	\ENDIF
\ENDWHILE
\smallskip \hrule \smallskip
\end{algorithmic}
\end{algorithm}

The highlighted part of the algorithm insures that it terminates for an input for which Algorithm \ref{alg:Buchberger} terminates.
Note that the rank $r$ (recall: $G$ and $S$ are contained in $(R*\Pi)^r\times R$) may grow as the algorithm progresses. 
