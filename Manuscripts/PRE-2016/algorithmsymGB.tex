  %  File:  hillarlevineAMS.tex

\documentclass{amsart}
\usepackage{amssymb}
 
\newtheorem{theorem}{Theorem}[section]
\newtheorem{lemma}[theorem]{Lemma}
\newtheorem{proposition}[theorem]{Proposition}
\newtheorem{corollary}[theorem]{Corollary}
\newtheorem{question}[theorem]{Question}\newtheorem{conjecture}[theorem]{Conjecture}

\theoremstyle{definition}
\newtheorem{definition}[theorem]{Definition}
\newtheorem{example}[theorem]{Example}
\newtheorem{xca}[theorem]{Exercise}
\newtheorem{algorithm}[theorem]{Algorithm}

\theoremstyle{remark}
\newtheorem{remark}[theorem]{Remark}

\numberwithin{equation}{section}

%    Absolute value notation
\newcommand{\abs}[1]{\lvert#1\rvert}

%    Blank box placeholder for figures (to avoid requiring any
%    particular graphics capabilities for printing this document).
\newcommand{\blankbox}[2]{%
  \parbox{\columnwidth}{\centering
%    Set 
fboxsep to 0 so that the actual size of the box will match the
%    given measurements more closely.
    \setlength{\fboxsep}{0pt}%
    
\fbox{\raisebox{0pt}[#2]{\hspace{#1}}}%
  }%
}

\newcommand{\floor}[1]{\lfloor {#1} \rfloor}
\newcommand{\ceil}[1]{\lceil {#1} \rceil}
\newcommand{\binomial}[2]{\left( 
\begin{array}{c} {#1} \\
                        {#2} \end{array} \right)}
\def\la{\langle}
\def\ra{\rangle}

\newcommand{\lm}{\operatorname{lm}}
\newcommand{\lc}{\operatorname{lc}}
\newcommand{\lt}{\operatorname{lt}}


\newcommand{\rank}{\textit{\rm rank}}
\newcommand{\<}{\langle}
\renewcommand{\>}{\rangle}


\DeclareSymbolFont{AMSb}{U}{msb}{m}{n}
\DeclareMathSymbol{\F}{\mathbin}{AMSb}{"46}
\DeclareMathSymbol{\N}{\mathbin}{AMSb}{"4E}
\DeclareMathSymbol{\Z}{\mathbin}{AMSb}{"5A}
\DeclareMathSymbol{\R}{\mathbin}{AMSb}{"52}
\DeclareMathSymbol{\C}{\mathbin}{AMSb}{"43}


\begin{document} \title[An Algorithms for Finding Symmetric Gr\"obner Bases]
{An Algorithm for Finding \\ Symmetric Gr\"obner Bases}

 %   Information for first author 
% \author{Christopher J. Hillar}
%    Address of record for the research reported here
%\address{Department of Mathematics, Texas A\&M University, College Station, TX 77843}
%\email{chillar@math.tamu.edu}

 
 \thanks{The work of the first author is supported under a National Science 
 Foundation Graduate Research Fellowship.} 

\subjclass{13E05, 13E15, 20B30, 06A07}%
\keywords{Invariant ideal, well-quasi-ordering, symmetric group, Gr\"obner basis, minimal generators}%


% \keywords{}

% ----------------------------------------------------------------
\begin{abstract}
We give an explicit algorithm to find Gr\"obner bases for symmetric
ideals in infinite dimensional polynomials rings.  This allows for symbolic computation
in a new class of rings.
\end{abstract} 

\maketitle 

\section{Introduction}
Let $X = \{x_1,x_2,\ldots\}$ be an infinite collection of
indeterminates, indexed by the positive integers, and let ${\mathfrak S}_{\infty}$ be the group of
permutations of $X$.  For a positive integer $N$, we will also let 
${\mathfrak S}_N$ denote the set of permutations of $\{1,\ldots,N\}$.
Fix a field $K$ and let $R = K[X]$ be the polynomial ring in the indeterminates $X$.  The group
${\mathfrak S}_{\infty}$ acts naturally on $R$: if $\sigma \in {\mathfrak
  S}_{\infty}$ and $f\in K[x_1,\dots,x_n]$, then 
\begin{equation}\label{groupaction}
\sigma f(x_1,\ldots,x_n) = f( x_{\sigma1},\dots,  x_{\sigma n})\in R.
\end{equation}
We let $R[{\mathfrak S}_{\infty}]$ be the (left) group ring of ${\mathfrak S}_{\infty}$ over $R$ 
with multiplication given by $f\sigma\cdot g\tau = fg(\sigma\tau)$ for $f,g\in R$ and
$\sigma,\tau\in {\mathfrak S}_{\infty}$, and extended by linearity.
The action (\ref{groupaction}) naturally gives $R$ the structure of a (left) module over the
ring $R[{\mathfrak S}_{\infty}]$.
An ideal $I \subseteq R$ is called \textit{invariant under ${\mathfrak S}_{\infty}$}
(or simply \textit{invariant}) if \[ {\mathfrak S}_{\infty}I := \{\sigma f
: \sigma \in {\mathfrak S}_{\infty}, \ f \in I\} \subseteq I.\] 
Invariant ideals are then simply the $R[{\mathfrak S}_{\infty}]$-submodules of
$R$.  

The following was proved recently in \cite{AH}.  It says that while ideals of $R$
are too big in general, those with extra structure have finite presentations.

\begin{theorem}\label{onevarfinitegenthm}
Every invariant ideal of $R$ is finitely generated as an $R[ {\mathfrak S}_{\infty}]$-module.  
In other words, $R$ is a Noetherian $R[{\mathfrak S}_{\infty}]$-module.
\end{theorem}


For the purposes of this work, we will use the following notation.
Let $B$ be a ring and let $G$ be a subset of a $B$-module $M$.  Then
$\<f: f \in G \>_{B}$ will denote the $B$-submodule of $M$ generated
by the elements of $G$.


\begin{example}
$I = \<x_1,x_2,\ldots\>_R$ is an invariant ideal of $R$.  Written as 
a module over the group ring $R[{\mathfrak S}_{\infty}]$, it has the
compact presentation $I= \<x_1\>_{R[{\mathfrak S}_{\infty}]}$.
\end{example}

Theorem \ref{onevarfinitegenthm} was motivated by finiteness questions in
chemistry \cite{Ruch1,Ruch2, Ruch3} and algebraic statistics \cite{SturmSull} 
involving chains of invariant ideals $I_k$ ($k = 1,2,\ldots$)
contained in finite dimensional polynomial rings $R_k$. 
We refer the reader to \cite{AH} for more details.

In the course of proving Theorem \ref{onevarfinitegenthm}, it was shown that,
in a certain sense, an invariant ideal $I$ has a finite
minimal Gr\"obner basis (see Section \ref{gbinfinite} for a review of these
concepts).  Moreover, the existence of such a set of generators solves
the ideal membership problem in $R$.  

\begin{theorem}
Let $G$ be a Gr\"obner basis for an invariant ideal $I$.  Then $f \in I$ 
if and only if $f$ has normal form $0$ with respect to $G$.
\end{theorem}

\begin{example}\label{trunccounterex}
Let $I = \<x_1 + x_2, x_1 x_2\>_{R[{\mathfrak S}_{\infty}]}$.  Then, a
Gr\"obner basis for $I$ is given by $G = \{x_1\}$.  It is important to note that we may not 
simply restrict consideration to $K[x_1,x_2]$ to produce this result since
%\[  \<x_1 + x_2, x_1 x_2\>_{R[{\mathfrak S}_{\infty}]} = \<x_1\>_{R[{\mathfrak S}_{\infty}]}\]
%but 
\[  \<x_1 + x_2, x_1 x_2\>_{R[{\mathfrak S}_{2}]} \neq  \<x_1\>_{R[{\mathfrak S}_{2}]}.\]
\end{example}

\begin{example}
The ideal $I = \<x_1^3 x_3 + x_1^2 x_2^3, 
x_2^2 x_3^2 - x_2^2 x_1 + x_1 x_3^2\>_{R[{\mathfrak S}_{\infty}]}$ has a 
Gr\"obner basis given by:
\[ G = {\mathfrak S}_{3} \cdot \{x_3 x_2 x_1^2, x_3^2 x_1 + x_2^4 x_1 - x_2^2 x_1, 
x_3 x_1^3, x_2 x_1^4, x_2^2 x_1^2\}.\]
Once $G$ is found, testing whether a polynomial $f$ is in $I$ is computationally fast.
\qed
\end{example}

The normal form reduction we are talking about here is a modification of the
standard notion in polynomial theory and Gr\"obner bases;  
we describe it in more detail in Section \ref{gbinfinite}.
Unfortunately, the techniques used to
prove finiteness in \cite{AH} are nonconstructive and therefore do not
give methods for computing Gr\"obner bases in $R$.
Our main result is an algorithm for finding these bases.

\begin{theorem}
Let $I = \<f_1, \ldots, f_n\>_{R[{\mathfrak S}_{\infty}]}$ be an invariant ideal of $R$.
There exists an effective algorithm to compute a finite minimal Gr\"obner basis for $I$.
\end{theorem}

\begin{corollary}
There exists an effective algorithm to solve the ideal membership problem
for symmetric ideals in the infinite dimensional ring $K[x_1,x_2,\ldots]$.
\end{corollary}

%In Section \ref{gbinfinite}, we describe an explicit family of finite, 
%minimal Gr\"obner bases.  This will be the content of Theorem \ref{monGBprop} below.

%If one picks even a single polynomial $g \in R$, the cyclic submodule
%$R[{\mathfrak S}_{X}] \cdot g$ is very large, and it is not clear
%that every submodule of $R$ doesn't arise in this way.  Given a finite
%list of polynomials $f_1,\ldots,f_k$, one could conceivably choose a sufficiently 
%large enough $N$ so that the number of unknowns in a system
%\[ f _i = \sum_{\sigma \in  {\mathfrak S}_N} r_{i \sigma} \sigma g,  \  \ \  r_{i \sigma} \in R, \ i = 1,\ldots,k,\]
%greatly outnumbers the number of equations, thereby (presumably) ensuring 
%a solution for the $r_{i \sigma}$.  

A brief review of 
the terminology and results of \cite{AH} is found in Section \ref{descriptionalgorithm},
including a new characterization (Theorem \ref{symordercharac}) of an important partial order on
monomials introduced by the authors of \cite{AH}.
Using this characterization, an explicit description of minimal 
Gr\"obner bases for monomial submodules is given by Thereom \ref{monGBprop}.

%\begin{example}
%We give two polynomials $f_1,f_2$ for which it is not obvious 
%that they could be generated by one element $g$, yet there does
%exists such a generator.
%\end{example}

In Section \ref{algorithm}, we describe our algorithm, and examples from an implementation
can be found in the subsequent section.  %Section \ref{algexamples}.
Finally, we prove correctness in Section \ref{proofcorrect}.


\section{Gr\"obner Bases for Invariant Ideals}\label{gbinfinite}

We first note that an infinite permutation acting on
a polynomial may be replaced with a finite one.
\begin{lemma}\label{infpermfiniteperm}
Let $\sigma \in {\mathfrak S}_\infty$ and $f \in R$.  Then there
exists a positive integer $N$ and $\tau \in {\mathfrak S}_N$ such that 
$\tau f = \sigma f$.
\end{lemma}
\begin{proof}
Let $S$ be the set of indices appearing in the monomials of $f$ and let 
$N$ be the largest integer in $\sigma S \cup S$.  The injective function $\sigma: S \to \{1,\ldots,N\}$
extends (nonuniquely) to a permutation $\tau \in  {\mathfrak S}_N$ such that $\tau f = \sigma f$.
\end{proof}


The following is a brief review of the Gr\"obner basis theory for invariant
ideals necessary we will need (see \cite{AH} for more details).


Let $\Omega$ be the set of monomials in indeterminates $x_1, x_2, \ldots$, 
including the constant monomial $1$.  Order the variables $x_1 < x_2 < \cdots$, and let 
$\leq$ be the induced lexicographic (total) well-ordering of monomials.  Given
a polynomial $f \in R$, we set lm$(f)$ to be the leading monomial of $f$ with
respect to $\leq$ and $\lt(f)$ to be its leading term.
The following partial ordering on $\Omega$ respects the action of ${\mathfrak S}_\infty$
and refines the division partial order on $\Omega$.

\begin{definition}\label{defpartialord}(The symmetric cancellation partial ordering)
\[v \preceq w \quad
:\Longleftrightarrow \quad \begin{cases} &\text{\parbox{200pt}{$v \leq
      w$ and there exist $\sigma \in {\mathfrak S}_\infty$ such that $\sigma v|w$ 
      and $\sigma u \leq \sigma v$ for all $u \leq v$.}}\end{cases}\]
\end{definition}

\begin{remark} A permutation $\sigma$ in the definition need
not be unique.  Also, we say that such a permutation \textit{witnesses} $v\preceq w$.
We will give a more computationally useful description of this partial order
in Theorem \ref{symordercharac} below.
\end{remark}

\begin{example}
As an example of this relation, consider the following
chain, \[x_1^3 \preceq x_1^2x_2^3 \preceq x_1^{\phantom{2}} x_2^{2} x_3^3.\]  To
verify the first inequality, notice that $x_1^2 x_2^3 = x_1^2 \sigma
(x_1^3)$, in which $\sigma$ is the transposition $(12)$.  If $u =
x_1^{u_1}\cdots x_n^{u_n} \leq x_1^3$, then it follows that $n =
1$ and $u_1 \leq 3$.  In particular, $\sigma u = x_2^{u_1}
\leq x_2^3  = \sigma x_1^3$.  Verification of the other inequality is similar.  

Alternatively, one may use Lemmas \ref{oneshiftuplem},  \ref{twoshiftuplem}, and
\ref{addtoendlem} to produce these and many other examples of such relations.  \qed
\end{example}

Although this partial order appears technical, it can be reconstructed from
the following two properties.  The first one says that the leading monomial 
of $\sigma f$ is the same as $\sigma \text{lm}(f)$ whenever there is a witness
$\sigma$ for lm$(f)$, while the latter can be viewed as a kind of  ``$S$-pair" leading term cancellation.

\begin{lemma}\label{cancellation}
Let $f$ be a nonzero polynomial and $w\in \Omega$.  Suppose that
$\sigma\in {\mathfrak S}_\infty$ witnesses $\text{\rm lm}(f)\preceq w$, and let $u\in \Omega$
with $u\sigma \text{\rm lm}(f)=w$. Then $\text{\rm lm}(u\sigma f)=u\sigma\text{\rm lm}(f)$.
\end{lemma}

\begin{lemma}
Suppose that $m_1 \preceq m_2$ and $f_1, f_2$ are two polynomials with 
lexicographic leading monomials $m_1$ and $m_2$, respectively.  
Then there exists a permutation $\sigma$
and $0 \neq c \in K$ such that \[ h = f_2 - c \frac{m_2}{\sigma m_1} \sigma f_1\] 
consists of monomials (lexicographically) smaller than $m_2$.
\end{lemma}

The following two lemmas allow us to generate many relations, including the
ones in the above example.  Proofs can also be found in \cite{AH}.

\begin{lemma}\label{oneshiftuplem}
Suppose that $x_1^{a_1}\cdots x_n^{a_n} \preceq x_1^{b_1}\cdots
x_n^{b_n}$ where $a_i,b_j\in\N$, $b_n>0$. Then for any $c \in \N$, we
have $x_1^{a_1}\cdots x_n^{a_n} \preceq x_1^c x_2^{b_1}\cdots
x_{n+1}^{b_n}$.
\end{lemma}

\begin{lemma}\label{twoshiftuplem}
  Suppose that $x_1^{a_1}\cdots x_n^{a_n} \preceq x_1^{b_1}\cdots x_n^{b_n}$,
  where $a_i,b_j\in\N$, $b_n>0$.   Then for any $a,b \in \N$ such that $a
  \leq b$, we have  $x_1^ax_2^{a_1}\cdots x_{n+1}^{a_n} \preceq
  x_1^bx_2^{b_1}\cdots x_{n+1}^{b_{n}}$.
\end{lemma}

The next fact is essentially a consequence of \cite[Lemma 2.14]{AH},
but we include an argument for completeness.  

\begin{lemma}\label{addtoendlem}
Let $u,v \in \Omega$ and set $n$ to be the largest index of indeterminates
appearing in $v$.  If $u \preceq v$, then there is a witness $\sigma \in {\mathfrak S}_n$,
and if $a,b \in \N$ are such that $a \leq b$, then $u x_{n+1}^{a} \preceq v x_{n+1}^{b}$.
\end{lemma}
\begin{proof}
Let $m$ (resp. $n$) be the largest integer such that $x_m | u$ (resp. $x_n | v$) 
and let $\sigma$ be a witness to $u \preceq v$.  
We first claim that $\sigma x_i \leq  x_n$ for all $i \leq m$.  To see this, suppose 
by way of contradiction that $\sigma x_i > x_n$ for some $i \leq m$. 
We have $\sigma u|v$, so if $x_i | u$, then
$\sigma x_i | v$, contradicting $\sigma x_i>x_n$; in particular, $x_i \neq x_m$. 
Assume now that $x_i<x_m$ so that $x_i < u$ and thus $\sigma x_i < \sigma u \leq  v$. 
Again this contradicts $\sigma x_i> x_n$ and finishes the proof of the claim.

It follows that $\sigma$ restricted to the set $\{x_i : i \leq m\}$ can be extended to a 
permutation $\sigma'$ of
$\{x_i : i \leq n\}$.  Furthermore, extending $\sigma'$ to a permutation in ${\mathfrak S}_\infty$
by setting $\sigma' x_i=x_i$ for all $i > n$, it is easy to see that 
$\sigma'$ still witnesses $u \preceq v$. 
The second claim in the lemma follows immediately from the first.
\end{proof}

In this setting, we need a notion of the leading monomials of a
set of polynomials that interacts with the symmetric group action.
For a set of polynomials $I$, we define
\[ \text{lm}(I) = \< w \in \Omega :  \text{lm}(f) \preceq w, \ 0 \neq f \in I\>_K,\]
the span of all monomials which are $\preceq$ larger than leading monomials
in $I$.  If $I$ happens to be an invariant ideal, then it follows from
Lemma \ref{cancellation} that \[\text{lm}(I) = \<\text{lm}(f): f \in I \>_K\] corresponds
to a more familiar set of monomials.  With these preliminaries in
place, we state the following definition from \cite{AH}.

\begin{definition}
We say that a subset $B$ of an invariant
ideal $I \subseteq R$ is a \emph{Gr\"obner basis}\/ for $I$
if lm$(B) = \text{lm}(I)$.
\end{definition}

Additionally, a Gr\"obner basis is called \textit{minimal} if
no leading monomial of an element in $B$ is $\preceq$ smaller
than any other leading monomial of an element in $B$.
%We should remark that the original notion of Gr\"obner basis was
%first introduced by Buchberger \cite{BB1,BB2} for the case of
%finitely many indeterminates $X$.
In analogy to the classical case, a Gr\"obner basis $B$
generates the ideal $I$:
\[ I =  \<B\>_{R[{\mathfrak S}_\infty]}.\]
The authors of \cite{AH} prove the following finiteness
result for invariant ideals; it is an analog to the corresponding
statement for finite dimensional polynomial rings.  As a corollary,
they obtain Theorem \ref{onevarfinitegenthm}.

\begin{theorem}
An invariant ideal of $R$ has a finite Gr\"obner basis.
\end{theorem}

Although much of the intuition involving Gr\"obner bases from the finite dimensional case 
transfers over faithfully to the ring $R$, one needs to be somewhat careful in general.  
For example, monomial generators do not automatically form a Gr\"obner basis for
an invariant ideal $I$ (see Example \ref{surpriseGBex} below).
However, we do have a description of minimal Gr\"obner bases for
monomial ideals, and this is the content Theorem \ref{monGBprop} below.  
To state it, we need to introduce a special class of permutations to give a 
more workable description of the symmetric cancellation partial order.

Fix a monomial $g = \mathbf{x}^{\mathbf{a}} = x_1^{a_1}\cdots x_n^{a_n}$. 
A \textit{downward elementary shift} (resp. \textit{upward elementary shift})
%of $g$ is a transposition $\sigma$ which acts
of $g$ is a permutation $\sigma$ which acts
on $\mathbf{a}$ as transposition of two consecutive coordinates, the smaller (resp. larger)
of which is zero.  A \textit{downward shift} (resp. \textit{upward shift})
of $g$ is a product of downward elementary shifts (resp. upward elementary shifts) that begin
with $g$. A \textit{shift permutation} of $g$ is either a downward shift or an upward shift of $g$.
If $g,h \in \Omega$ and $\sigma$ is an upward shift of $g$ with $h = \sigma g$, then we
write $g \sim_{\sigma} h$.  
%For example, $\sigma = (34)$ is an upward elementary 
%shift of $g =  x_2^3x_3x_5^2$ and  $\tau = (32)(56)$(34) is an upward shift of $g$; 
%in this case, $g \sim_{\tau} h$ for $h = x_3^3x_4x_6^2$.
For example, $\sigma = (341)$ is an upward elementary 
shift of $g =  x_2^3x_3^{\phantom{3}}x_5^2$ and  $\tau = (32)(56)(341)$ is an upward shift of $g$; 
in this case, $g \sim_{\tau} h$ for $h = x_3^3x_4^{\phantom{3}}x_6^2$.

The following fact should be clear.

\begin{lemma}\label{translem}
If $g \sim_{\sigma} h$ and $h \sim_{\tau} k$, then $g \sim_{\tau \sigma} k$.
\end{lemma}

A more concrete description of these permutations is given by the
following straightforward lemma, which follows directly from the
definitions.
 
\begin{lemma}\label{shiftcharac}
Let $g$ be a monomial, and let $i_1 < \cdots < i_n$ be those 
indices appearing in the indeterminates dividing $g$.
Then $\sigma$ is an upward shift permutation of $g$ if and only if 
\[\sigma i_1 < \sigma i_2 <  \cdots < \sigma i_n \ \ \text{and} \ \  \sigma i_k \geq i_k, \ \ \ k = 1,\ldots,n.\]
\end{lemma}

The following fact gives a relationship between shift permutations and 
the symmetric cancellation partial order.

\begin{lemma}\label{invshiftlem}
Let $g$ and $h$ be monomials with $g \sim_{\sigma} h$ for some $\sigma \in {\mathfrak S}_\infty$. 
Then $g \preceq h$.  Moreover, we have $h \sim_{\sigma^{-1}} g$.
\end{lemma}
\begin{proof}
By transitivity and Lemma \ref{translem}, we may 
suppose that $\sigma$ as in the statement of the lemma acts on $g$ by
transposing $x_i$ and $x_{i+1}$.  
Write $g = x_1^{a_1} \cdots x_i^{a_i}x_{i+2}^{a_{i+2}} \cdots x_n^{a_n}$ with
$a_n > 0$; we must verify that 
\[ x_1^{a_1} \cdots x_i^{a_i}x_{i+2}^{a_{i+2}} \cdots x_n^{a_n}
\preceq x_1^{a_1} \cdots x_{i-1}^{a_{i-1}}x_{i+1}^{a_i} x_{i+2}^{a_{i+2}} \cdots x_n^{a_n}.\]
%so that $h = x_1^{a_1} \cdots x_{n-1}^{a_{n-1}}x_{n+1}^{a_n} x_{n+2}^{a_{n+2}} \cdots x_m^{a_m}$.
This is proved by induction on $n$.  When $n=1$, we have $i =1$, and the
claim reduces to Lemma \ref{oneshiftuplem}.  In general, we have two cases to consider.
If $i = n > 1$, then the claim follows from Lemma \ref{twoshiftuplem} and induction.
Alternatively, if $i < n$ and $n > 1$, then we may apply Lemma \ref{addtoendlem} and 
induction.  The second claim is clear from the definitions.
\end{proof}

\begin{remark}\label{carefulwitness}
A word of caution is in order.  Suppose that $g$ and $h$ are monomials with 
$g \sim_{\sigma} h$ for some  $\sigma \in {\mathfrak S}_\infty$.  Then it can happen that
$\sigma$ is \textit{not} a witness for the (valid) relation $g \preceq h$.  For example, if $\sigma = (14)(23)$, 
$g = x_2$, and $h = x_3$, then $g \sim_{\sigma} h$.  However, the relation $x_1 \leq x_2$
does not imply $\sigma x_1 \leq \sigma x_2$ as one can easily check.
\end{remark}


We now state and prove a characterization of the symmetric cancellation partial order.

%[****** The statement of the theorem is wrong...it needs to be adjusted*****]

\begin{theorem}\label{symordercharac}
Two monomials $v$ and $w$ satisfy  $v \preceq w$ if and only if
there is an upward shift $\sigma \in {\mathfrak S}_N$ of $v$ such that
$\sigma v | w$, where $N$ is the largest index of indeterminates appearing in $w$.
\end{theorem}
\begin{proof}
We prove the only-if direction ($\Rightarrow$);
the converse is clear from Lemma \ref{invshiftlem} and Definition \ref{defpartialord}.
Let $N$ be the largest index of indeterminates appearing in $w$.
If $v \preceq w$, then there is a monomial $m$ and a witness $\sigma \in {\mathfrak S}_N$
such that $w = m\sigma v$ by Lemma \ref{addtoendlem}.  
For the rest of the argument, we fix this permutation $\sigma$.
We will prove that $\sigma$ is an upward shift of $v$ using the 
characterization found in Lemma \ref{shiftcharac}.

Write $v = x_{i_1}^{v_{i_1}} \cdots x_{i_n}^{v_{i_n}}$, in 
which $i_1 < \cdots < i_n$ are all the indices appearing in $v$.
%If $n = 1$ and $i_1 = 1$, then the claim is clear, so we suppose from now on
%that $i_n > 1$.  
We prove the following claim by induction on the number of 
indeterminates $n$ appearing in $v$:
\begin{equation}\label{maininduct}
(u \leq v \Rightarrow  \sigma u \leq \sigma v  
\text{ for all } u \in \Omega) \Rightarrow ( \sigma i_{1} < \cdots < \sigma i_{n} \text{ and }
  i_k \leq \sigma i_k \text{ for all } k \leq n).
\end{equation}
The result in the theorem is then implied by Lemma \ref{shiftcharac}.
We take for our base case of induction $n = 0$ (so that $v = 1$),
as the statement is vacuously true.
Also, if $n = 1$ and $i_1 = 1$, then the statement is clear, 
so we suppose from now on that $i_n > 1$. 

Fix a monomial $v$ with $n+1$ indeterminates; we must show that 
(\ref{maininduct}) holds.  Therefore, assume that $\sigma$ is such that
$u \leq v \Rightarrow  \sigma u \leq \sigma v  
\text{ for all } u \in \Omega$.  For a positive integer $c$, consider the monomial 
$u_c =  (x_{1} \cdots x_{i_{n+1}-1})^c \leq v$.  Since $u_c \leq v$,
we have by assumption that
\[  \sigma u = (x_{\sigma 1} \cdots x_{\sigma (i_{n+1}-1)})^c  \leq 
x_{\sigma i_1}^{v_{i_1}} \cdots x_{\sigma i_{n+1}}^{v_{i_{n+1}}} = \sigma v.\]
If $\sigma i_{n+1} \leq \sigma i_j$ for some $j < n+1$, then
by choosing $c$ sufficiently large (say, larger than
the degree of $v$), 
the above inequality is impossible.
Therefore, it follows that  $\sigma i_j < \sigma i_{n+1}$ for all $j < n+1$.  Next, 
we show that $i_{n+1} \leq \sigma i_{n+1}$.  Suppose by way of contradiction
that $\sigma  i_{n+1} < i_{n+1}$.  Then, $\sigma  i_j < i_{n+1}$ for all $j < n+1$.
In particular, $\sigma v < v$, and thus $\sigma^s v \leq \sigma v < v$ for
all positive integers $s$.  Hence, $v = \sigma^{N!} v < v$, a contradiction.

Our final step is to invoke the induction hypothesis and prove the 
other inequalities on the right-hand side of (\ref{maininduct}).  Suppose that 
$u = x_{1}^{u_1} \cdots x_{i_{n}}^{{u_{i_{n}}}} \leq x_{i_1}^{v_{i_1}} 
\cdots x_{i_{n}}^{{v_{i_{n}}}}$ so that $u  x_{i_{n+1}}^{v_{i_{n+1}}} \leq v$.  
By assumption, we have 
\[  \sigma (u  x_{i_{n+1}}^{v_{i_{n+1}}}) =  (\sigma u)  x_{\sigma i_{n+1}}^{v_{i_{n+1}}} \leq 
x_{\sigma i_{k}}^{v_{i_{k}}} \cdots x_{\sigma i_{n+1}}^{v_{i_{n+1}}} = \sigma v,\]
and thus (since we are using the lexicographic ordering),
\[   \sigma u   \leq  x_{\sigma i_{1}}^{v_{i_{1}}} \cdots x_{\sigma i_{n}}^{v_{i_{n}}}.\]
It follows from induction applied to the monomial $x_{i_1}^{v_{i_1}} \cdots x_{i_{n}}^{{v_{i_{n}}}}$
in $n$ indeterminates that $\sigma i_{1} < \cdots < \sigma i_{n} \text{ and }
i_k \leq \sigma i_k \text{ for all } k \leq n$.  This proves the claim and completes the 
proof of the theorem.
\end{proof}

We may now prove the main result of this section.

\begin{theorem}\label{monGBprop}
Let $G$ be a set of $n$ monomials 
of degree $d$, and let $N$ be the largest index of indeterminates appearing
in any monomial in $G$.  Then $H = {\mathfrak S}_N G$ is a 
(finite) Gr\"obner basis for $I = \<G\>_{R[{\mathfrak S}_\infty]}$.
Moreover, if we let
\[S =\{ h \in H :  \text{ there exists $g  \in H \backslash \{h\}$ and  $\sigma \in 
{\mathfrak S}_N$ with $g \sim_{\sigma} h$}\},\] 
then $H \backslash S$ is a minimal Gr\"obner basis 
for $I$.
\end{theorem}
\begin{proof}
Let $G$, $H$, $S$, $N$, and $I$ be as in the statement of the theorem;  we first show that $H$ is a 
Gr\"obner basis for $I$. The inclusion $\text{lm}(H) \subseteq \text{lm}(I)$ is clear 
from the definition.  So suppose that $w \in \text{lm}(I)$ is a monomial; we must  show that 
$h \preceq w$ for some $h \in H$.  Set $w = u \sigma g$ for some monomial $u$, witness 
$\sigma \in {\mathfrak S}_\infty$, and $g \in G$.  Since  $\sigma g \preceq u\sigma g = w$,
it suffices to show that $h \preceq \sigma g$ for some $h \in H$.
Let $\tau$ be a downward shift that takes $\sigma g$ to a monomial $h$
with indices at most $N$.  Then $h$ has the same type as $g$, and
therefore there is a permutation $\gamma \in {\mathfrak S}_N$ such that
$h = \gamma g $. It follows that $h \in H$ and $h \sim_{\tau^{-1}} \sigma g$
so that $h \preceq \sigma g$ by Lemma \ref{invshiftlem}.

Next, we observe that $H \backslash S$ is still a Gr\"obner 
basis since $g \sim_{\sigma} h$ implies that $g \preceq h$.
Therefore, it remains to prove that $H \backslash S$ is minimal.
If $h,g \in H$ are related by $g \preceq h$, then $h = m \sigma g$ for 
a witness $\sigma$ and a monomial $m$.  Since each element of $H$
has the same degree, we have $m = 1$.  By Theorem \ref{symordercharac},
it follows that we may choose $\sigma \in {\mathfrak S}_N$ such that
$g \sim_{\sigma} h$.  Therefore, we are only removing unnecessary
elements from the Gr\"obner basis $H$ when we discard the monomials in
$S$. This completes the proof.
\end{proof}

\begin{corollary}
Let $G$ be a finite set of monomials, and let $N$ be the largest
index of indeterminates appearing in any monomial in $G$.  Then  ${\mathfrak S}_N G$
is a (not necessarily minimal) Gr\"obner basis for $I = \<G\>_{R[{\mathfrak S}_\infty]}$.
\end{corollary}

\begin{example}\label{surpriseGBex}
%The ideal $I = \<x_1^2 x_2^{\phantom{3}}\>_{R[{\mathfrak S}_\infty]}$ has a minimal Gr\"obner
%basis $G = \{x_1^2 x_2, x_1 x_2^2\}$.  Neither element of $G$ can be
%removed because although $x_1^{\phantom{3}} x_2^2 = (12)(x_1^2 x_2^{\phantom{3}})$, 
%we do not have $x_1^{\phantom{3}} x_2^2 \preceq x_1^2 x_2^{\phantom{3}}$ or 
%$x_1^{\phantom{3}} x_2^2 \preceq x_1^2 x_2^{\phantom{3}}$, as one
%can readily verify. \qed
The ideal $I = \<x_1^2 x_3^{\phantom{3}}\>_{R[{\mathfrak S}_\infty]}$ has a Gr\"obner
basis, \[H = \{x_1^{\phantom{3}} x_2^2, x_1^{\phantom{3}} x_3^2, x_1^2 x_2^{\phantom{3}}, 
x_2^{\phantom{3}} x_3^2, x_1^2 x_3^{\phantom{3}},x_2^2 x_3^{\phantom{3}}\}.\] 
However, it is not minimal.  Removing those elements that are the result 
of upward shifts, we are left with the following minimal Gr\"obner basis for $I$:  
$\{x_1^{\phantom{3}} x_2^2, x_1^2x_2^{\phantom{3}}\}$. 
\qed
\end{example}


\section{Reduction of polynomials}\label{descriptionalgorithm}

Before describing the algorithm, we must recall the ideas of reduction from
\cite{AH}.  Let $f\in R$, $f\neq 0$, and let $B$ be a set of nonzero polynomials
in $R$. We say that $f$ is \emph{reducible by $B$}\/ if there exist
pairwise distinct $g_1,\dots,g_m\in B$, $m\geq 1$, such that for each
$i$ we have $\lm(g_i)\preceq \lm(f)$, witnessed by some $\sigma_i\in
G$, and
$$\lt(f) = a_1w_1\sigma_1\lt(g_1) + \cdots + a_mw_m\sigma_m\lt(g_m)$$
for nonzero $a_i\in A$ and monomials $w_i\in X^\diamond$ such that
$w_i\sigma_i\lm(g_i)=\lm(f)$.  In this case we write
$f\underset{B}\longrightarrow h$, where
$$h=f - \big(a_1w_1\sigma_1g_1 + \cdots + a_mw_m\sigma_mg_m\big),$$
and we say that $f$ \emph{reduces to $h$} by $B$.  We say that $f$ is
\emph{reduced} with respect to $B$ if $f$ is not reducible by $B$. By
convention, the zero polynomial is reduced with respect to $B$.
Trivially, every element of $B$ reduces to $0$.




The smallest quasi-ordering on $R$ extending the relation
$\underset{B}\longrightarrow$ is denoted by
$\underset{B}{\overset{*}\longrightarrow}$.  If $f,h\neq 0$ and
$f\underset{B}\longrightarrow h$, then $\lm(h)<\lm(f)$, by
Lemma~\ref{cancellation}.  In particular, every chain
$$h_0\underset{B}\longrightarrow h_1\underset{B}\longrightarrow h_2
\underset{B}\longrightarrow \cdots$$
with all $h_i\in R\setminus\{0\}$
is finite. (Since the term ordering $\leq$ is well-founded.) Hence
there exists $r\in R$ such that
$f\underset{B}{\overset{*}\longrightarrow} r$ and $r$ is reduced with
respect to $B$; we call such an $r$ a \emph{normal form} of $f$ with
respect to $B$.

\begin{lemma}\label{reduction}
  Suppose that $f\underset{B}{\overset{*}\longrightarrow} r$. Then
  there exist $g_1,\dots,g_n\in B$, $\sigma_1,\dots,\sigma_n\in G$ and
  $h_1,\dots,h_n\in R$ such that
  $$f=r+\sum_{i=1}^n h_i\sigma_i g_i\quad \text{and}\quad \lm(f)\geq
  \max_{1\leq i\leq n}\lm(h_i\sigma_ig_i).$$
  \textup{(}In particular,
  $f-r\in \<B\>_{R[G]}$.\textup{)}
\end{lemma}
\begin{proof}
See \cite{AH}.
\end{proof}
%\begin{proof}
%  This is clear if $f=r$. Otherwise we have $f
%  \underset{B}\longrightarrow h
%  \underset{B}{\overset{*}\longrightarrow} r$ for some $h\in R$.
%  Inductively we may assume that there exist $g_1,\dots,g_n\in B$,
%  $\sigma_1,\dots,\sigma_n\in G$
%such that $\lm(g_i)\preceq\lm(f)$ for each $i$, witnessed by $\sigma_i$,
%  and $h_1,\dots,h_n\in R$ such that
%  $$h=r+\sum_{i=1}^n h_i\sigma_i g_i\quad \text{and}\quad \lm(h)\geq
%  \max_{1\leq i\leq n}\lm(h_i\sigma_ig_i).$$
%  There are also
%  $g_{n+1},\dots,g_{n+m}\in B$, $\sigma_{n+1},\dots,\sigma_{n+m}\in
%  G$, $a_{n+1},\dots,a_{n+m}\in A$ and $w_{n+1},\dots,w_{n+m}\in
%  X^\diamond$ such that $\lm(w_{n+i}\sigma_{n+i}g_{n+i})=\lm(f)$ for
%  all $i$ and
%\[ \lt(f) = \sum_{i=1}^m a_{n+i}w_{n+i}\sigma_{n+i}\lt(g_{n+i}), \qquad f
%= h + \sum_{i=1}^m a_{n+i} w_{n+i}\sigma_{n+i} g_{n+i}.\] Hence
%putting $h_{n+i}:=a_{n+i}w_{n+i}$ for $i=1,\dots,m$ we have
%$f=r+\sum_{j=1}^{n+m} h_j\sigma_j g_j$ and $\lm(f)>\lm(h)\geq
%\lm(h_j\sigma_jg_j)$ if $1\leq j\leq n$, $\lm(f)=\lm(h_j\sigma_jg_j)$
%if $n<j\leq n+m$.
%\end{proof}

\begin{lemma}\label{char GB}
  Let $I$ be an invariant ideal of $R$ and $B$ be a set of nonzero
  elements of $I$. The following are equivalent:
\begin{enumerate}
\item $B$ is a Gr\"obner basis for $I$.
\item Every nonzero $f\in I$ is reducible by $B$.
\item Every $f\in I$ has normal form $0$. \textup{(}In particular,
  $I=\<B\>_{R[G]}$.\textup{)}
\item Every $f\in I$ has unique normal form $0$.
\end{enumerate}
\end{lemma}
\begin{proof}
  The implications
  (1)~$\Rightarrow$~(2)~$\Rightarrow$~(3)~$\Rightarrow$~(4) are either
  obvious or follow from the remarks preceding the lemma.  Suppose
  that (4) holds. Every $f\in I\setminus\{0\}$ with
  $\lt(f)\notin\lt(B)$ is reduced with respect to $B$, hence has two
  distinct normal forms ($0$ and $f$), a contradiction. Thus
  $\lt(I)=\lt(B)$.
\end{proof}


\section{Description of the Algorithm}\label{algorithm}


We begin by describing a method that checks when two monomials are $\preceq$ comparable,
returning a permutation (if it exists) witnessing the relation.
This is accomplished using the characterization given by Theorem \ref{symordercharac}.
In this regard, it will be useful to view monomials in $R$ as vectors of 
integers $v = (v_1,v_2,\ldots)$ with finite support in $\mathbb N^{\omega}$.

\begin{algorithm}\label{vwcompalg}\mbox{}(Comparing monomials in the symmetric
cancellation order)\\
Input: Two monomials $v$ and $w$ with largest indeterminate in $w$ being $N$.\\
Output: A permutation $\sigma \in {\mathfrak S}_N$ if $v \preceq w$; otherwise, \textbf{\emph{false}}.
\begin{enumerate}
\item Set $t := 1$, $match := \{\}$;
\item For $i = 1$ to N:
\item[]\hspace{0.5cm} For $j = t$ to N:
\item[]\hspace{1.0cm} If $v_i \neq 0$ and $v_i \leq w_j$, then 
\item[]\hspace{1.5cm} $t := j+1$;
\item[]\hspace{1.5cm} $match := match \cup \{(i,j)\}$;
\item[]\hspace{1.5cm} Break inner loop;
\item[]\hspace{0.5cm} $t := \max\{i+1,t\}$;

\item If $match$ contains fewer elements than the support of $v$, return \textbf{\emph{false}};

\item For $j = N$ down to $1$:
\item[]\hspace{0.5cm} Set $i :=$ largest integer not appearing as a first coordinate in $match$;

\item[]\hspace{0.5cm} If $j$ is not a second coordinate in $match$, then $match := match \cup (i,j)$;
\item Return the permutation that $match$ represents;
\end{enumerate}
\end{algorithm}

\begin{remark}
One must be somewhat careful when constructing the witness $\sigma$.  Changing the recipe
given in the algorithm above might produce incorrect results.  See also Remark \ref{carefulwitness}.
\end{remark}

%In words, the algorithm is to find out whether $v$ is a substring of $w$ 
%(excluding $0$s in $v$) with the additional constraint that 

\begin{example}
Consider the vectors $v = (1,2,0,2)$ and $w = (0,3,4,1)$ representing monomials
$x_4^2 x_2^2 x_1$ and $x_4 x_3^4 x_2^3$ respectively.  
Then, Algorithm \ref{vwcompalg} will return false since $match = \{(1,2),(2,3)\}$ 
contains less than three elements after Step $(2)$.

On the other hand, running the algorithm on inputs $v = (3,2,0,0,5)$ and
$w = (5,1,4,6,9)$ will produce an output of $\{(1,1),(2,3),(3,2),(4,4),(5,5)\}$, which
correctly gives the witness $\sigma = (23)$ to the relation 
$x_1^3 x_2^2 x_5^5 \preceq x_1^5 x_2x_3^4 x_4^6 x_5^9$.
\end{example}

We also need to know how to compute a reduction of a polynomial $f$ by another 
polynomial $g$ (assuming that $f$ is reducible by $g$).  Given a witness $\sigma$, however,
this is calculated in Lemma \ref{cancellation}.  Specifically, we set
\begin{equation}\label{sgpoly}
SG_{\sigma}(f,g) = f - \frac{\lt(f)}{\sigma \lt(g)} \sigma g.
\end{equation}

Notice that when $\sigma = (1)$, the polynomial $SG_{\sigma}(f,g)$ is 
the normal $S$-pair from standard Gr\"obner basis theory.

The general case of reducing
a polynomial $f$ by a set $B$ is performed as follows; it is a modification of
ordinary polynomial division in the setting of finite dimensional polynomial rings.

\begin{algorithm}\label{reducefbyg}\mbox{}(Reducing a polynomial $f$ by an ordered  set of polynomials $B$)\\
Input: Polynomial $f$ and an ordered set $B = (b_1,\ldots,b_s) \in R^s$.\\
Output: The reduction of $f$ by $B$. 
\begin{enumerate}
\item Set $p := f$, $r := 0$, $divoccured := 0$;
\item While $p \neq 0$:
\item[]\hspace{0.5cm} i := 1;
\item[]\hspace{0.5cm} $divoccured := 0$;

\item[]\hspace{0.5cm} While $i \leq s$;
\item[]\hspace{1.0cm} $g := b_i$;
\item[]\hspace{1.0cm} If there exists a $\sigma$ witnessing $\lm(g) \preceq \lm(p)$, then
\item[]\hspace{1.5cm} $p := SG_{\sigma}(p,g)$;
\item[]\hspace{1.5cm} $divoccured := 1$;
\item[]\hspace{1.5cm} Break inner loop;
\item[]\hspace{1.0cm} Else, $i := i + 1$;
\item[]\hspace{0.5cm} If $divoccured = 0$, then 
\item[]\hspace{1.0cm} $r := r + \lt(p)$;
\item[]\hspace{1.0cm} $p := p - \lt(p)$;
\item Return $r$;
\end{enumerate}
\end{algorithm}

\begin{example}
Let $f = x_3^2 x_2^2+x_2 x_1$ and $B = (x_3 x_1+x_2 x_1)$.  Reducing $f$ by
$B$ is the same as reducing $f$ by $x_3 x_1+x_2 x_1$ twice as one can check.  The
resulting polynomial is $x_2^3 x_1 + x_2 x_1$.  
\end{example}

Before coming to our main result, we describe a truncated version of it.

\begin{algorithm}\label{symGBtruncalg}\mbox{}(Constructing a truncated Gr\"obner basis for a symmetric ideal)\\
Input: An integer $N$ and polynomials $F = \{f_1,\ldots,f_n\} \subset  K[x_1,\ldots,x_N]$.\\
Output: A truncated Gr\"obner basis for $I = \<f_1, \ldots, f_n\>_{R[{\mathfrak S}_{\infty}]}$. 
\begin{enumerate}
\item Set $F' := F$;
\item For each pair $(f_i,f_j)$:
\item[]\hspace{0.5cm} For each pair $(\sigma, \tau)$ of permutations in ${\mathfrak S}_{N}$:

\item[]\hspace{1.0cm} $h := SG_{(1)}(\sigma f_i, \tau f_j)$;
\item[]\hspace{1.0cm} Set $r$ to be the reduction of $h$ by ${\mathfrak S}_{N}B'$; 
\item[]\hspace{1.0cm} If $r \neq 0$, then $B' := B' \cup \{r\}$;

\item Return $B'$;
\end{enumerate}
\end{algorithm}

\begin{remark}
As we have seen, it is not enough to choose $N$ to be the largest indeterminate
appearing in $F$ (c.f. Remark \ref{trunccounterex}). 
\end{remark}

We call the input $N$ the \textit{order} of a truncated basis for $F$.

\begin{algorithm}\label{symGBalg}\mbox{}(Constructing a Gr\"obner basis for a symmetric ideal)\\
Input: Polynomials $F = \{f_1,\ldots,f_n\} \subset  K[x_1,\ldots,x_N]$.\\
Output: A Gr\"obner basis for $I = \<f_1, \ldots, f_n\>_{R[{\mathfrak S}_{\infty}]}$. 
\begin{enumerate}
\item Set $F' := F$, $i := N$;
\item While true:
\item[]\hspace{0.5cm} Set $F'$ to be a truncated Gr\"obner basis of $F$ of order $i$;
\item[]\hspace{0.5cm} If every element of $F'$ reduces to $0$ by ${\mathfrak S}_{N}F$, then return $F$;
\item[]\hspace{0.5cm} $F := F'$;
\item[]\hspace{0.5cm} $i := i + 1$;
\end{enumerate}
\end{algorithm}



We postpone the proof of correctness of the algorithms above until Section \ref{proofcorrect}

\section{Examples}\label{algexamples}

Here we list some examples of our algorithm.\footnote{Code that performs the calculations in this section using SINGULAR 3.0 (http://www.singular.uni-kl.de) can be found at http://www.math.tamu.edu/$\sim$chillar/.}

Consider $F = \{x_1+x_2, x_1 x_2\}$ from the introduction.  One iteration of 
Algorithm \ref{symGBalg} with $i = 2$ gives $F' = \{x_1+x_2, x_1^2\}$.
The next two iterations produce $\{x_1\}$ and thus the algorithm returns with this 
as its asnwer. 

\section{Proof of Correctness}\label{proofcorrect}

Here we prove that our algorithm terminates and 
produces a Gr\"obner basis for an ideal $I$.




% ----------------------------------------------------------------
\begin{thebibliography}{99}

\bibitem{AH}
M. Aschenbrenner and C. Hillar, \emph{Finite generation of symmetric ideals}, 
Trans. Amer. Math. Soc., to appear.

%\bibitem{BB1} B. Buchberger, \emph{An algorithm for finding a basis
%for the residue class ring of a zero-dimensional polynomial
%ideal}, PhD Thesis, University of Innsbruck,
%Institute for Mathematics (1965).

%\bibitem{BB2} B. Buchberger, \emph{An algorithmic criterion for the
%solvability of algebraic systems of equations},
%Aequat. Math. \textbf{4} (1970), 374--383.

\bibitem{Cox2} D. Cox, J. Little, D. O'Shea, \emph{Using algebraic geometry},
Springer, New York, 1998.

\bibitem{HW} 
C. Hillar and T. Windfeldt, \emph{Minimal generators for symmetric ideals}, preprint.

\bibitem{Ruch1} A. Mead, E. Ruch, A. Sch\"onhofer, \emph{Theory of
    chirality functions, generalized for molecules with chiral
    ligands}.  Theor. Chim.  Acta \textbf{29} (1973), 269--304.

\bibitem{Ruch2} E. Ruch, A. Sch\"onhofer, \emph{Theorie der
    Chiralit\"atsfunktionen}, Theor. Chim. Acta \textbf{19} (1970),
  225--287.

\bibitem{Ruch3} E. Ruch, A. Sch\"onhofer, I. Ugi, \emph{Die
    Vandermondesche Determinante als N\"aherungsansatz f\"ur eine
    Chiralit\"atsbeobachtung, ihre Verwendung in der Stereochemie und
    zur Berechnung der optischen Aktivit\"at}, Theor. Chim.  Acta
  \textbf{7} (1967), 420--432.

\bibitem{JS}
J. Schicho, private communication, 2006.

\bibitem{SturmSull}
B. Sturmfels and S. Sullivant, \emph{Algebraic factor analysis: tetrads, 
pentads and beyond}, preprint. (math.ST/0509390).


\end{thebibliography}

\end{document}
% ----------------------------------------------------------------

