  %  File:  EGB_algorithm.tex

\documentclass{amsart}
\usepackage{amssymb}
\usepackage{algorithmic}
 
\newtheorem{theorem}{Theorem}[section]
\newtheorem{lemma}[theorem]{Lemma}
\newtheorem{proposition}[theorem]{Proposition}
\newtheorem{corollary}[theorem]{Corollary}
\newtheorem{question}[theorem]{Question}\newtheorem{conjecture}[theorem]{Conjecture}

\theoremstyle{definition}
\newtheorem{definition}[theorem]{Definition}
\newtheorem{example}[theorem]{Example}
\newtheorem{problem}[theorem]{Problem}
\newtheorem{xca}[theorem]{Exercise}
\newtheorem{algorithm}[theorem]{Algorithm}

\theoremstyle{remark}
\newtheorem{remark}[theorem]{Remark}
\newtheorem{criterion}[theorem]{Criterion}

\numberwithin{equation}{section}

\newcommand{\B}[1]{\mathbb #1}
\newcommand{\C}[1]{\mathcal #1}
\newcommand{\F}[1]{\mathfrak #1}

%    Absolute value notation
\newcommand{\abs}[1]{\lvert#1\rvert}

%    Blank box placeholder for figures (to avoid requiring any
%    particular graphics capabilities for printing this document).
\newcommand{\blankbox}[2]{%
  \parbox{\columnwidth}{\centering
%    Set 
fboxsep to 0 so that the actual size of the box will match the
%    given measurements more closely.
    \setlength{\fboxsep}{0pt}%
    
\fbox{\raisebox{0pt}[#2]{\hspace{#1}}}%
  }%
}

\newcommand{\floor}[1]{\lfloor {#1} \rfloor}
\newcommand{\ceil}[1]{\lceil {#1} \rceil}
\newcommand{\binomial}[2]{\left( 
\begin{array}{c} {#1} \\
                        {#2} \end{array} \right)}
\def\la{\langle}
\def\ra{\rangle}

\newcommand{\lm}{\operatorname{lm}}
\newcommand{\lc}{\operatorname{lc}}
\newcommand{\lt}{\operatorname{lt}}


\newcommand{\rank}{\textit{\rm rank}}
\newcommand{\<}{\langle}
\renewcommand{\>}{\rangle}
\newcommand{\ideal}[1]{\langle #1 \rangle}
\newcommand{\LT}{\operatorname{in}_>}
\newcommand{\Inc}{\operatorname{Inc}(\B N)}
\newcommand{\NF}{\operatorname{NF}}


\begin{document} \title[Algorithms for Equivariant Gr\"obner Bases]
{Algorithms for Equivariant Gr\"obner Bases}

 %   Information for first author 
\author{Christopher J. Hillar}
\address{Redwood Center for Theoretical Neuroscience, University of California, Berkeley}
\email{chillar@msri.org}

\author{Robert Krone}
\address{Georgia Institute of Technology, Atlanta, GA}
\email{krone@math.gatech.edu}

\author{Anton Leykin}
\address{Georgia Institute of Technology, Atlanta, GA}
\email{leykin@math.gatech.edu}

%\thanks{The work of the first author is supported under a National Science Foundation Graduate Research Fellowship.} 

\subjclass{13E05, 13E15, 20B30, 06A07}%

% \keywords{Invariant ideal, well-quasi-ordering, symmetric group, Gr\"obner basis, generating sets}


% \keywords{}

% ----------------------------------------------------------------
\maketitle

\section{Introduction}
Let $X = \{x_1,x_2,\ldots\}$ be an countably infinite collection of indeterminates.  Fixing a field $k$, let $R = k[X]$ be the polynomial ring with over $k$ with indeterminates $X$.  Let $G$ be a semigroup with a left action on $X$, so there is a natural left action of $G$ on $R$.  For a polynomial $f \in R$ and semigroup element $\sigma \in G$, the action of $g$ on $f$ is defined as
 \[ \sigma\cdot f(x_1,x_2,\ldots) = f(\sigma(x_1),\sigma(x_2),\ldots). \]

Indexing $X$ by the natural numbers, two semigroups of particular interest are
\begin{itemize}
 \item ${\mathfrak S}_{\infty}$, the group of all permutations of $\B N$, and
 \item $\Inc$, the semigroup of all strictly increasing functions $\B N \to \B N$.
\end{itemize}
Other semigroup actions of interest come from indexing the variables in other ways:
\begin{itemize}
 \item Index $X$ by $\B N \times [n]$ for some positive integer $n$ and act with $\Inc$ on only the first index.
 \item Index $X$ by $\B N \times \B N$ and act with $\Inc$ diagonally on both indices.
\end{itemize}


The left actions of $G$ and $R$ on $R$ can be combined into an action of the twisted semigroup ring of $G$ over $R$, denoted $R*G$.  The additive structure of $R*G$ is the same as the semigroup ring $R[G]$.  Multiplication is defined term-wise by 
 \[ f\sigma\cdot g\tau = f\sigma(g)(\sigma\tau) \]
for $f,g\in R$ and $\sigma,\tau\in G$, and extended by linearity.  Note that elements of $R$ and $G$ do not commute with each other in $R*G$, mirroring the lack of commutativity of acting on $R$ by permuting the variables and by multiplying by a polynomial.  $R$ has a natural $R*G$-module structure.

\begin{definition}
 An ideal $I \subseteq R$ is $G$\textit{-equivariant}
(or simply \textit{equivariant}) if \[ GI := \{\sigma f
: \sigma \in G, \ f \in I\} \subseteq I.\]
\end{definition}
$G$-equivariant ideals are exactly the $R*G$-submodules of $R$.

\begin{definition}
$R$ is $G${\em -Noetherian} if it satisfies the ascending chain condition for $G$-equivariant ideals. 
\end{definition}
In particular $G$-Noetherianity implies every $G$-equivariant ideal is finitely generated as an $R*G$-module, which is of particular interest to us.  The notation $\ideal{f_1,\ldots,f_s}_{R*G}$ will be used to denote the equivariant ideal generated by polynomials $f_1,\ldots,f_s$ as an $R*G$-module.

\begin{theorem}\label{onevarfinitegenthm}
$R$ is $\F S_\infty$-Noetherian.  $R$ is $\Inc$-Noetherian.
\end{theorem}

\begin{example}
$I = \<x_1,x_2,\ldots\>_R$ is a $\F S_\infty$-equivariant ideal of $R$.  It can be expressed as $I= \<x_1\>_{R*\F S_\infty}$.
\end{example}

\section{Equivariant Gr\"obner Bases}
In order to define Gr\"obner bases, we give $R$ a monomial order $>$, and impose the following requirement on the relationship between $G$ and the order:
\begin{itemize}
 \item For any monomials $x^\alpha, x^\beta \in R$, and $\sigma \in G$,
\[ x^\alpha > x^\beta \quad \Leftrightarrow \quad \sigma x^\alpha > \sigma x^\beta. \]
\end{itemize}
If this condition is met we say $G$ respects the order $>$.
Note that there is no monomial order which the action of $\F S_\infty$ respects so we won't be able to define $\F S_\infty$-equivariant Gr\"obner bases.  However there are monomial orders which respect $\Inc$, for example lexicographic order with
\[x_1 < x_2 < x_3 < \cdots.\]
We can use $\Inc$ as a substitute for $\F S_\infty$ using the following fact.
\begin{theorem}
For any finite $F \subset R$, there exists $n$ such that $F \subset k[x_1,\ldots,x_n]$.  Then
 \[ \ideal{F}_{\F S_\infty} = \ideal{\F S_n F}_{\Inc}. \]
\end{theorem}
So any $\F S_\infty$-equivariant ideal generated by $F$ can be represented as a $\Inc$-equivariant ideal with a finite generating set easily constructed from $F$.


\begin{definition}
 A $G$-equivariant Gr\"obner basis for equivariant ideal $I$ with monomial order $>$ that respects $G$ is a set $B \subset I$ such that for any $f \in I$, there is $g \in B$ such that
  \[ \LT f = m\cdot \LT g \]
 for some monomial $m \in R*G$. 
\end{definition}

The normal form of a polynomial $f$ with respect to a set of polynomials $B$, denoted $\NF_B(f)$, is defined in the usual way.
\begin{algorithm}
$NormalForm$ \\
Inputs: finite set $B \subset R$ and polynomial $f \in R$.\\
Outputs: the normal form $\NF_B(f)$.
\begin{algorithmic}
\STATE $divisionOccurred :=$ \TRUE ;
\WHILE{$divisionOccurred$ \AND $f \neq 0$}
	\STATE $divisionOccurred =$ \FALSE ;
	\FOR{$g \in B$}
		\IF{there is $m \in R*G$ such that $\LT f = m \LT g$}
			\STATE $f = f - mg;$
			\STATE $divisionOccurred =$ \TRUE ;
			\STATE break;
		\ENDIF
	\ENDFOR
\ENDWHILE
\RETURN $f;$
\end{algorithmic}
\end{algorithm}

If $B$ is a Gr\"obner basis of ideal $\ideal{B}_{R*G}$ then $\NF_B(f)$ is well-defined, but not for general sets $B$ as it may depend on the choice of $g \in B$ to reduce by at each step.

A difficulty with this equivariant normal form algorithm is in finding $\sigma \in G$ such that $\sigma \LT g$ divides $\LT f$.  This can be difficult depending on the choice of semigroup action.  In the case of $G = \Inc$ acting on a single index, there is a linear time greedy algorithm, by mapping each variable in $\LT g$ in turn to the first possible variable in $\LT f$ with a large enough exponent.  For $G = \Inc$ acting diagonally on variables indexed by $\B N \times \B N$, we are not aware of a polynomial time algorithm.  Computing $\sigma$ as efficiently as possible is an opportunity for improvement.

\begin{theorem}
Let $B$ be a $G$-equivariant Gr\"obner basis for equivariant ideal $I$.  Then $f \in I$ if and only if $f$ has normal form $0$ with respect to $B$.
\end{theorem}

\section{Equivariant Buchberger algorithm}
If $R$ is $G$-Noetherian, then every equivariant ideal has a finite equivariant Gr\"obner basis.  Without $G$-Noetherianity certain finitely generated ideals may still have a finite equivariant Gr\"obner basis, but this is not guaranteed.  In either case, given a set of generators for an equivariant ideal, we can try to use a variant of the Buchberger algorithm to compute a Gr\"obner basis.

\begin{algorithm}
$EquivariantBuchberger$ \\
Inputs: generators $F = \{f_1,\ldots,f_s\}$ of equivariant ideal $I$.\\
Outputs: equivariant Gr\"obner basis $B$ of $I$.
\begin{algorithmic}
\STATE $B := F;$
\STATE $newB := $ \TRUE;
\WHILE{$newB$}
	\STATE $newB = $ \FALSE;
	\FORALL{$(f,g) \in B \times B$}
		\STATE $Spolynomials := FindSpolynomials(f,g);$
		\FORALL{$s \in Spolynomials$}
			\IF{$\NF_B(s) \neq 0$}
				\STATE append $\NF_B(s)$ to $B;$
				\STATE $newB = $ \TRUE;
			\ENDIF
		\ENDFOR
	\ENDFOR
\ENDWHILE
\RETURN $B;$
\end{algorithmic}
\end{algorithm}

This follows the structure of the traditional Buchberger algorithm, but two important points of departure take place in the computation of $\NF_B$ (as discussed above) and $FindSpolynomials$.  In the usual Buchberger algorithm, for each pair $f,g \in B$ there is only one S-polynomial to consider
\[ S(f,g) := \frac{m}{\LT f}f - \frac{m}{\LT g}g \]
where $m = \gcd(\LT f, \LT g)$.  For any monomials $m_1,m_2 \in R$ such that $\LT m_1f = \LT m_2g$, the difference $m_1f - m_2g$ is divisible by $S(f,g)$.  However expanding the choice of monomials to $m_1,m_2 \in R*G$ there is no longer a single polynomial which generates all differences $m_1f - m_2g$ where $\LT m_1f = \LT m_2g$.  Instead we need to generate all S-polynomials in the set
 \[ S(Gf,Gg) := \{ S(\sigma f, \tau g): \sigma,\tau \in G\}. \]
Typically this set is infinite, so for the algorithm to succeed we need to impose another requirement on the action of $G$:
\begin{itemize}
 \item For each $f,g \in R$, the set $Gf \times Gg$ is contained in the union of of a finite number of $G$-orbits $G(\sigma_1 f, \tau_1 g),\ldots,G(\sigma_r f, \tau_r g)$, and we can compute the pairs $(\sigma_1,\tau_1),\ldots,(\sigma_r,\tau_r)$.
\end{itemize}

\begin{theorem}
 Fixing $f,g \in R$, there is some $n$ such that $f,g \in k[x_1,\ldots,x_n]$.  Then for any $\sigma, \tau \in \Inc$, there exist strictly increasing functions $\sigma',\tau': [n] \to [2n]$ and $\rho \in \Inc$ such that
  \[ \rho (\sigma' f, \tau' g) = (\sigma f, \tau g). \]
\end{theorem}

Therefore we can consider only S-polynomials of the form $S(\sigma' f,\tau' g)$.  The same theorem also holds for $\Inc$ acting on variables indexed by $\B N \times \B N$ or $\B N \times [n]$.  Note that the number of pairs of strictly increasing functions $[n] \to [2n]$ is $\binom{2n}{n}^2$, which is large but finite.  We can make some improvements on the number of S-polynomials considered for each pair $f,g \in B$, but this remains a major bottleneck in the run time of the algorithm.








\end{document}
% ----------------------------------------------------------------

